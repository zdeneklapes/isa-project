
%%%%%%%%%%%%%%%%%%%%%%%%%%%%%%%%%%%%%%%
%%%%%%%%%%%%%%%%%%%%%%%%%%%%%%%% Návod na použití %%%%%%%%%%%%%%%%%%%%%%%%%%
\section{Návod na použití} \label{sec:usage}

\subsection{Překlad}\label{section:compilation}

Překlad je prováděn pomocí aplikace \opustt{GNU Make}.
Soubor s definicí překladu se nacházi v~souboru \opustt{Makefile}.
Podporované argumenty pro přikaz \opustt{make} jsou:

\begin{itemize}
    \item \opustt{make} Prelozi obe aplikace
    \item \opustt{make all} Přeloží obě aplikace
    \item \opustt{make sender} Přeloží pouze aplikaci \opustt{dns\_sender}
    \item \opustt{make receiver} Přeloží pouze aplikaci \opustt{dns\_receiver}
    \item \opustt{make pack} Zabalí projekt
\end{itemize}

Obě aplikace se po překladu nachází v~kořenovém adresáři projektu.

\subsection{Spouštění programu \opustt{dns\_sender}}

\opustt{dns\_sender [-u UPSTREAM\_DNS\_IP] {BASE\_HOST} {DST\_FILEPATH} [SRC\_FILEPATH]} \\

\textit{Přepínače:}

\begin{itemize}
    \item \opustt{-u <UPSTREAM\_DNS\_IP>} Slouží k vynucení vzdáleného DNS serveru
\end{itemize}

\textit{Poziční parametry:}

\begin{itemize}
    \item \opustt{BASE\_HOST} Slouží k nastavení bázové domény všech přenosů
    \item \opustt{DST\_FILEPATH} Cesta pod kterou se data uloží na serveru
    \item \opustt{SRC\_FILEPATH} Cesta k souboru který bude odesílán, pokud není specifikováno pak program čte data ze STDIN
\end{itemize}

Povinné parametry jsou \opustt{BASE\_HOST}, \opustt{DST\_FILEPATH}.

Při chybně zadaných parametrech se vypíše napověda programu. Nápovědu
je možné si vypsat také příkazem \opustt{./dns\_sender -h}.

\subsection{Spouštění programu \opustt{dns\_receiver}}

\opustt{dns\_receiver {BASE\_HOST} {DST\_DIRPATH}} \\

\textit{Poziční parametry:}

\begin{itemize}
    \item \opustt{BASE\_HOST} Slouží k nastavení bázové domény všech přenosů
    \item \opustt{DST\_DIRPATH} Cesta pod kterou se budou všechny příchozí data/soubory ukládat
\end{itemize}

Povinné parametry jsou \opustt{BASE\_HOST}, \opustt{DST\_DIRPATH}.

Při chybně zadaných parametrech se vypíše napověda programu. Nápovědu
si je možné vypsat také příkazem \opustt{./dns\_receiver -h}.
