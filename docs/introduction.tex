\section{Úvod}
\label{sec:uvod}
Cílem projektu bylo vytvořit aplikaci pro tunelování datových přenosů pomocí
DNS dotazů v~jazyce C\@. Aplikace by měla umožnit přenášet data mezi klientem
a serverem.
Klient posílá DNS dotazy se zakodovanými daty do qname na DNS server,
který dotazy zpracovavá, ukláda do souboru a~posílá odpověď klientovi.
Aplikace by měla umožnit přenášet data ze souboru a standardního vstupu.

Oba programy (\texttt{dns\_sender} a \texttt{dns\_receiver}) je možné
ovládat přes vstupní argumenty programu, viz\ref{sec:usage}.

%%%%%%%%%%%%%%%%%%%%%%%%%%%%%%%% Popis DNS Tunelovaní %%%%%%%%%%%%%%%%%%%%%


\section{Popis mechanismu pro tunelování datových přenosů prostřednictvím DNS dotazů}
\label{sec:popis-mechanismu-pro-tunelovani-datovych-prenosu-prostrednictvim-dns-dotazu}

Domain name system, DNS je protokol překládající url adresu na ip adresu.
DNS dotaz je zpráva, která se posílá na DNS server, který odpoví na dotaz.
DNS dotaz je složen z hlavičky a těla dotazu\cite{dnsPacket}.

Tunelování pomocí DNS dotazu využívají utočníci, kteří
mají na svém serveru nainstalovaný malware.
Klientův dotaz je směrován na utočníkům server, kterému to umožní
nainstalovat malware na klientovi a provádět další akce.
