\section{Popis návrhu a implementace klientské a serverové aplikace}
\label{sec:popis-navrhu-a-implementace-klientske-a-serverove-aplikace}
Adresářová struktura:

Adresářová struktura je rozdělena do 3 adresářů.

\begin{itemize}
    \item \texttt{sender} \- obsahuje zdrojové soubory pro klienta
    \item \texttt{receiver} \- obsahuje zdrojové soboury pro serveru
    \item \texttt{common} \- obsahuje zdrojové soubory, které jsou využívány\\
    obema programy \texttt{dns\_sender} a \texttt{dns\_receiver}
    \item \texttt{middleman} \- obsahuje kód pro otestování ztráty \\
    packetu (Pouze jako ukázka otestování, při běžném běhu aplikaci se nepoužívá).\\
    Makro \texttt{TEST\_LOSS\_PACKET} nacházející se v \texttt{./commmon/dns\_helper.h} zapne middlemana.
\end{itemize}

Oba programy jsou řízeny stavovým automatem, který rozlišuje stav
na základě typu paketu a podle
něho provadí další specifické kroky pro vytvaření dotazů a odpovědí.


%%%%%%%%%%%%%%%%%%%%%%%%%%%%%%%% Klient %%%%%%%%%%%%%%%%%%%%%%%%%%%%%%%%

\subsection{Klient \- dns\_sender}
Na začátku programu jsou alokovány a inicializovány celkem 3 struktury.
\begin{itemize}
    \item \texttt{program\_t} \- obsahuje všechny promměnné programu
    \item \texttt{dns\_datagram\_t} \- obsahuje všechny promměnné datagramu
    \item \texttt{args\_t} \- obsahuje všechny vstupní promměnné programu
\end{itemize}

\subsubsection{Parsovaní argumentu} \label{sec:parsovani-argumentu-k}
Vstupní argumenty klienta jsou parsovaný pomocí vlastní algoritmické
implementace, kvůli několika druhům argumentů
(povinné poziční, povinné nepoziční, nepovinné poziční).

Parsování zajišťuje funkce \texttt{void set\_args\_sender(program\_t *program)} \\
v souboru \texttt{common/argument\_parser.c}.

\subsubsection{Vytváření paketu a čtení ze souboru} \label{sec:vytvareni-paketu-k}

Paket je vytvářen podle aktuálního typu packetu, kterých je celkem 5 druhů:

\paragraph{START}
Informační paket, následující paket bude obsahovat název cílového souboru.
Při tomto typu paketu se nastaví \texttt{program\->dgram\->packet\_type} na \texttt{FILENAME}.

V qname DNS dotazu je pouze \texttt{ZAKODOVANY\_STRING\_"START".BASEHOST}.

\paragraph{FILENAME}
Paket obsahující název cílového souboru.

Délka cílového souboru je omezena na 1024 znaku.

Soubor je postupně rozparsován a zakódován
do více paketů v případě, že není možné název souboru
zakódovat do jednoho qname DNS dotazu,
jehož maximální povolená délka je 255 znaků\cite{dnsPacketInfo}.

V qname DNS dotazu je \texttt{ZAKODOVANY\_FILENAME.BASEHOST}.

\paragraph{DATA}
Informační paket, že následující paket bude obsahovat data souboru.

V qname DNS dotazu je pouze \texttt{ZAKODOVANY\_STRING\_"DATA".BASEHOST}.

\paragraph{SENDING}
Paket obsahující data souboru/stdin.

Obsah souboru je postupně čten, rozparsován a zakodován do qname DNS dotazu.
Soubor se čte postupne podle možné délky dat pro
zakodování do qname, kterou počítám makrem nacházejícího se v souboru \\
\texttt{common/dns\_helper.c}: \\
\texttt{\#define BASE32\_LENGTH\_DECODE(src\_size) (ceil((src\_size) / 1.6))}. \\
Makro jako vstup bere maxímální délku po zakódování a vrácí délku před zakodováním.
V případě, že není možné data zakódovat do jednoho qname DNS dotazu,
tak je soubor rozdělen na více paketů.

V qname DNS dotazu je \texttt{ZAKODOVANY\_DATA.BASEHOST}.

\paragraph{END}
Informační koncový paket, že všechna data byla přenesena.

V qname DNS dotazu je \texttt{ZAKODOVANY\_STRING\_"END".BASEHOST}.

\subsubsection{Odeslani dotazu} \label{sec:odeslani-dotazu-k}
Dotaz je odeslán okamžitě při každém vytvoření paketu viz\ref{sec:vytvareni-paketu-k}.
Pro zakódování dat se používá funkce: \\
\texttt{int base32\_encode(const uint8\_t *data, int length, \\
uint8\_t *result, int bufSize)}\cite{encodingData}.

\subsubsection{Prijimani odpovedi}\label{sec:prijimani-odpovedi-k}
Klient očekává odpověď, pokud nedorazí do 5 sekund, tak klient odešle paket znova.
To je řešeno pomocí funkce následujícího volání funkce:
\texttt{setsockopt(dgram\->network\_info.socket\_fd, \\
    SOL\_SOCKET, SO\_RCVTIMEO, \&timeout, sizeof(timeout))},\\
které se provádí v souboru \texttt{common/initialization.c}.


%%%%%%%%%%%%%%%%%%%%%%%%%%%%%%%% Server %%%%%%%%%%%%%%%%%%%%%%%%%%%%%%%%

\subsection{Server \- dns\_receiver}

Na začátku programu jsou alokovány celkem 3 struktury.

\begin{itemize}
    \item \texttt{program\_t} \- obsahuje všechny promměnné programu
    \item \texttt{dns\_datagram\_t} \- obsahuje všechny promměnné datagramu
    \item \texttt{args\_t} \- obsahuje všechny vstupní promměnné programu
\end{itemize}

\subsubsection{Parsování argumentů} \label{sec:parsovani-argumentu-s}

Argumenty serveru jsou parsovány pomocí knihovní funkce \texttt{getopt}, \\
kvůli vypisování nápovědy. Všechny ostatní argumenty jsou poziční a povinné,
což zjednodušuje parsování vstupních argumentů programu.

Parsování zajišťuje funkce \texttt{void set\_args\_receiver(program\_t *program)} \\
v souboru \texttt{common/argument\_parser.c}.

\subsubsection{Parsování paketu a ukládání dat do souboru} \label{sec:vytvareni-paketu-s}
Paket je parsovaným podle aktualního typu packetu, kterých je serveru
rozlišujeme celkem 8 druhů:

\paragraph{START}
Informační paket, následující paket bude obsahovat název cílového souboru.

Při tomto typu paketu se nastaví \texttt{program\->dgram\->packet\_type} \\
na \texttt{FILENAME}.

\paragraph{FILENAME}
Paket obsahující název cílového souboru.

Délka cílového souboru je omezena na 1024 znaků.

Cílový soubor je parsován v nekonecné smyčce z qname DNS dotazu\\
a uložen do proměnné \texttt{program\->args\->filename}, dokud nepříjde\\
další informační paket s daty \texttt{DATA}.

\paragraph{DATA}
Informační paket, že následující paket bude obsahovat data souboru.

Při tomto typu paketu se nastaví \texttt{program\->dgram\->packet\_type} na \texttt{SENDING}.

\paragraph{SENDING}
Paket obsahující data souboru/stdin.

Data jsou z qname DNS dotazu parsována, dekódována a
zapisována do souboru v nekonečné smyčce, dokud nepřijde další
informační paket s daty \texttt{END}.

\paragraph{END}
Informační koncový paket, že data byla přenesena.

Při tomto typu paketu se nastaví \texttt{program\->dgram\->packet\_type} na \texttt{WAITING\_NEXT\_FILE}.

\paragraph{WAITING\_NEXT\_FILE}
Čeka se na prenos dalšího souboru.

\paragraph{RESEND\_OR\_BADBASEHOST\_\_AFTER\_FILENAME}
Chyba v base host, nebo je paket přenášen znovu, protože někde nastala chyba.
Do tohodle stavu se přejde, pokud aktuální stav je \texttt{FILENAME}.

\paragraph{RESEND\_OR\_BADBASEHOST\_\_AFTER\_SENDING}
Chyba v base host, nebo je paket přenášen znovu, protože někde nastala chyba.
Do tohodle stavu se přejde, pokud aktuální stav je \texttt{SENDING}.

\subsubsection{Příjem dotazu} \label{sec:odeslani-dotazu-s}
Při příjmu dat se pro dekódování používá funkce
\texttt{ int base32\_decode(const uint8\_t *encoded, uint8\_t *result, int bufSize) }\cite{encodingData}.

\subsubsection{Odeslani odpovedi}\label{sec:prijimani-odpovedi-s}
Po úspěšném zápisu dat do souboru se odešle odpověď na dotaz.
K DNS dotazu se přidá answer a paket se odešle zpět.
